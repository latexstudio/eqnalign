% \iffalse meta-comment
%!TeX program = T_do_dtx
% 
% This part is not going to appear anywhere so is good for 'header'
% information.
%
%
%<*driver>
\documentclass{ltxdoc}

\usepackage[vmargin=0.7in,hmargin={1.5in,0.6in},marginparsep=0.1in,marginparwidth=1in,a4paper]{geometry}
\usepackage{lmodern,anttor}
\usepackage[T1]{fontenc}
\usepackage{enumerate}
\usepackage{xcolor}
\usepackage{url, hyperref}

\DisableCrossrefs
\CodelineNumbered
\def\theCodelineNo{{\color{black}\normalfont\fontseries{sb}\selectfont\footnotesize\arabic{CodelineNo}}}
\def\MacroFont{\ttfamily\color{blue}}

% Whatever packages here
%
% Might include
% \usepackage{\jobname}
%
% Very handy is
% \usepackage[numbered]{hypdoc} % hyperref for l3doc
%
% You might want the following
%\EnableCrossrefs
%\CodelineIndex
%\RecordChanges
\begin{document}
  \DocInput{\jobname.dtx}
\end{document}
%</driver>
% \fi
%
% ^^A Comments in the documentation part start ^^A
%
% ^^A If you use the extracted package itself you can do
% \GetFileInfo{\jobname.sty}
% ^^A and define \filedate, etc.
%
% ^^A Meta data for the package
% \title{\pkg{mypkg}\thanks{^^A
%   This file describes version \fileversion, last revised \filedate.^^A
% }}
% \author{^^A
%   Tom Hejda\thanks{E-mail: tohecz@gmail.com}^^A
% }
% \date{Released \filedate}
%
% ^^A \section{User documentation}
%
% ^^A User documentation would appear here.
%
% ^^A This will appear at the end of the PDF, which might or might not include
% ^^A the typeset code
% \StopEventually{^^A
%   \PrintChanges
%   \PrintIndex
% }
%
% \vspace*{2cm}
% \begingroup
%  \centering
%  {\Huge\bfseries \LaTeX\@ package \texttt{eqnalign}\par}
%  \bigskip
%  {\Large\bfseries --- --- making |eqnarray|(|*|) look and work like |align|(|*|) --- ---\par}
%  \bigskip
%  {\Large by Tom Hejda, \texttt{tohecz@gmail.com}\par}
% \endgroup
% \vspace*{2cm}
%
% %%%%%%%%%%%%%%%%%%%%%%%%%%%%%%%%%%%%%%%%%%%%%%%%%%%%%%%%%
% \section{The goal}
% %%%%%%%%%%%%%%%%%%%%%%%%%%%%%%%%%%%%%%%%%%%%%%%%%%%%%%%%%
%
% The goal of this package is to allow easy conversion from the insanely-looking |eqnarray| to
%  the look and behaviour of |align| from |amsmath|.
% It is inspired by a \TeX.StackEchange question \url{http://tex.stackexchange.com/q/96210/11002} by a user called ``Werner'',
%  and by an answer of mine to the question.
%
% %%%%%%%%%%%%%%%%%%%%%%%%%%%%%%%%%%%%%%%%%%%%%%%%%%%%%%%%%
% \section{The behaviour}
% %%%%%%%%%%%%%%%%%%%%%%%%%%%%%%%%%%%%%%%%%%%%%%%%%%%%%%%%%
%
% The package is activated by simply loading it, and it does not have any package options.
% It just redefines |eqnarray| and |eqnarray*|,
%  and then it makes |amsthm| aware of this redefinition so that |\qedhere| works inside these environments.
%
% %%%%%%%%%%%%%%%%%%%%%%%%%%%%%%%%%%%%%%%%%%%%%%%%%%%%%%%%%
% \section{License and remarks}
% %%%%%%%%%%%%%%%%%%%%%%%%%%%%%%%%%%%%%%%%%%%%%%%%%%%%%%%%%
%
% \begin{enumerate}[(1)]
%
% \item The package is licensed under the \LaTeX\@ Project Public License version 1.3c (LPPL v1.3c) or higher.
% The latest version of this license is in \url{http://www.latex-project.org/lppl.txt}\,.
%
% \item Note that actually, the usage of this package is discouraged, in favour of converting the code into
%  proper ``|amsmath| code'', using the true |align| and |align*| environments.
% It is intended for cases where a lot of already existing code needs to be converted and there
%  is no capacity for doing it the right but time-consuming way.
%
% \item All bugs shall be reported to the GitHub page \url{http://github.com/tohecz/eqnalign}\,.
% Just note that unless the bug is crucial or easy to deal with, it may not be fixed since
%  (per the previous remark) single problematic cases shall be dealt with by other means.
%
% \end{enumerate}
%
% %%%%%%%%%%%%%%%%%%%%%%%%%%%%%%%%%%%%%%%%%%%%%%%%%%%%%%%%%
% \section{Implementation}
% %%%%%%%%%%%%%%%%%%%%%%%%%%%%%%%%%%%%%%%%%%%%%%%%%%%%%%%%%
%
% Note that we in general say |eqnarray| where we really mean either |eqnarray| or |eqnarray*|.
%
%    \begin{macrocode}
%<*package>
%    \end{macrocode}
% Package header.
%    \begin{macrocode}
\ProvidesPackage{eqnalign}[2016/06/09 v1.0 Make eqnarray(*) behave like align(*).]
%    \end{macrocode}
% The only necessary package is |amsmath| so that |align| and |align*| are defined.
%    \begin{macrocode}
\RequirePackage{amsmath}
%    \end{macrocode}
%
% The package does some catcode mysteries that shouldn't propagate out, so we make everything
%  in a group and use |\gdef| everywhere.
%    \begin{macrocode}
\begingroup
%    \end{macrocode}
%
% \begin{macro}{\eqna@origamp}
%    We store a catcode-4 (tab alignment char) |&| in a macro.
%    We need a catcode-13 (active) |&| througout the rest of the package.
%    \begin{macrocode}
\catcode`\&=4
\gdef\eqna@tab@amp{&}
\catcode`\&=13
%    \end{macrocode}
% \end{macro}
%
% \begin{macro}{\eqna@new@amp}
%    This will be the replacement of |&| inside |eqnarray|.
%    We use |\eqna@amp@| if the innermost environment is |eqnarray| and |\eqna@origamp| otherwise;
%     this is to allow things like arrays and matrices inside |eqnarray|.
%    \begin{macrocode}
\gdef\eqna@new@amp{%
	\ifx\@currenvir\eqna@currenvir 
		\eqna@amp@
	\else
		\eqna@tab@amp
	\fi
}
%    \end{macrocode}
% \end{macro}
%
% \begin{macro}{\eqna@amp@i}
% \begin{macro}{\eqna@amp@ii}
% \begin{macro}{\eqna@amp@iii}
%    Three macros that are ``rotated'', after the first, the second shall be used, then the third.
%    The third one ends a group since it ends a table cell,
%     therefore after that the first one is again in action.
%    The first |&| on a line is kept, the second is ignored, the third is kept.
%    \begin{macrocode}
\gdef\eqna@amp@i{\eqna@tab@amp\let\eqna@amp@\eqna@amp@ii}
\gdef\eqna@amp@ii{\let\eqna@amp@\eqna@amp@iii}
\gdef\eqna@amp@iii{\eqna@tab@amp}
%    \end{macrocode}
% \end{macro}
% \end{macro}
% \end{macro}
%
% \begin{macro}{\eqna@doamp}
%    The default is |\eqna@amp@i|.
%    \begin{macrocode}
\global\let\eqna@amp@\eqna@amp@i
%    \end{macrocode}
% \end{macro}
%
% \begin{macro}{\eqna@hook}
%    The default meaning of |&| is |\eqna@amp@i|.
%    We store the current environment, which is either |eqnarray| or |eqnarray*|;
%     it is used in |\eqna@new@amp| for the test for inner environments.
%    Then we activate |&| and make its meaning |\eqna@new@amp|.
%    \begin{macrocode}
\gdef\eqna@hook{%
	\let\eqna@currenvir\@currenvir
	\catcode`\&=\active
	\let&\eqna@new@amp
}
%    \end{macrocode}
% \end{macro}
%
% Now we will be defining environments containing |*| in name, so we make it a letter.
%    \begin{macrocode}
\catcode`\*=11
%    \end{macrocode}
%
% \begin{macro}{\eqna@def@env}
% \begin{macro}{eqnarray}
% \begin{macro}{eqnarray*}
%    We define a macro |\eqna@def@env| that contains the redefinitions of |eqnarray| (and |eqnarray*|).
%    The environments themselves are like |align|, just hooked using |\eqna@hook|.
%    We then call this macro immediately to define the environments.
%    (All this fuss with |\eqna@def@env| is to correct things in case |hyperref| is loaded after |eqnalign|.
%    \begin{macrocode}
\gdef\eqna@def@env{%
	\gdef\eqnarray{\eqna@hook\align}%
	\gdef\eqnarray*{\eqna@hook\align*}%
	\global\let\endeqnarray\endalign
	\global\let\endeqnarray*\endalign*
}
\eqna@def@env
%    \end{macrocode}
% \end{macro}
% \end{macro}
% \end{macro}
%
% \begin{macro}{\eqnarray@qed}
% \begin{macro}{\eqnarray*@qed}
%    To make |amsthm|'s |\qedhere| work in |eqnarray|, we need to ``hint'' |amsthm| that it exists
%    \begin{macrocode}
\global\let\eqnarray@qed\align@qed
\global\let\eqnarray*@qed\align*@qed
%    \end{macrocode}
% \end{macro}
% \end{macro}
%
% End the group we began at the very beginning.
%    \begin{macrocode}
\endgroup
%    \end{macrocode}
%
% \begin{macro}{EQNarray}
%    Just of sentiment, we keep the original |eqnarray| as |EQNarray|.
%    \begin{macrocode}
\def\EQNarray{%
	\stepcounter{equation}%
	\def\@currentlabel{\p@equation\theequation}%
	\global\@eqnswtrue
	\m@th
	\global\@eqcnt\z@
	\tabskip\@centering
	\let\\\@eqncr
	$$\everycr{}\halign to\displaywidth\bgroup
		 \hskip\@centering$\displaystyle\tabskip\z@skip{##}$\@eqnsel
		&\global\@eqcnt\@ne\hskip \tw@\arraycolsep \hfil${##}$\hfil
		&\global\@eqcnt\tw@ \hskip \tw@\arraycolsep
			$\displaystyle{##}$\hfil\tabskip\@centering
		&\global\@eqcnt\thr@@ \hb@xt@\z@\bgroup\hss##\egroup
			\tabskip\z@skip
		\cr
}
\def\endEQNarray{%
		\@@eqncr
		\egroup
		\global\advance\c@equation\m@ne
	$$\@ignoretrue
}
\@namedef{EQNarray*}{\def\@eqncr{\nonumber\@seqncr}\EQNarray}
\@namedef{endEQNarray*}{\nonumber\endEQNarray}
%    \end{macrocode}
% \end{macro}
%
% Last but not least, if |hyperref| is loaded after |eqnalign| (and only in that case),
%  we issue a warning since |hyperref| is doing bad things to |eqnarray|, and we redefine |eqnalign| once more.
%    \begin{macrocode}
\@ifpackageloaded{hyperref}{}{
	\AtBeginDocument{
		\@ifpackageloaded{hyperref}{
			\@latex@warning{Package `eqnalign' should be loaded after
			`hyperref'.\MessageBreak Redefining `eqnarray' and `eqnarray*' at this
			point \MessageBreak and crossing fingers...}
			\eqna@def@env
		}{}
	}
}
%    \end{macrocode}
%
% That's all.
%    \begin{macrocode}
\endinput
%    \end{macrocode}
%    \begin{macrocode}
%</package>
%    \end{macrocode}
%
% \Finale